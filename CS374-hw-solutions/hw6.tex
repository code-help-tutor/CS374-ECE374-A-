\documentclass[11pt]{article}
\usepackage{fullpage,amsmath,hyperref,graphicx}
\newcommand{\eps}{\varepsilon}
\newcommand{\fig}[2]{\begin{figure}[h]\begin{center}%
  \includegraphics[scale=#2]{#1.pdf}\end{center}%
  \end{figure}}

\begin{document}

\begin{center}\Large\bf 
CS/ECE 374 A (Spring 2022)\\
{\Large Homework 6} (due March 10 Thursday at 10am)
\end{center}

\medskip\noindent
{\bf Instructions:} As in previous homeworks.

\bigskip\noindent
{\bf Note:} In any dynamic programming solution, you should 
follow the steps below (if we explicitly state that pseudocode is not required, then step~4 may be skipped):
\begin{enumerate}
\item first give a clear, precise definition of the subproblems (i.e., what the recursive
function is intended to compute);
\item then derive a recursive formula to solve the subproblems (including
base cases), with justification or proof of correctness if the formula is not
obvious;
\item specify a valid evaluation order;
\item give pseudocode to evaluate your recursive formula bottom-up (with loops instead of recursion);
and 
\item analyze the running time.
\end{enumerate}
{\em Do not jump to pseudocode immediately.  Never skip step~1!}


\begin{description}
\bigskip
\item[Problem 6.1:] 
For a sequence $\langle b_1,\ldots,b_m\rangle$, an \emph{alternation} is an index $i\in\{2,\ldots,m-1\}$
such that ($b_{i-1}<b_i$ and $b_i>b_{i+1}$) or ($b_{i-1}>b_i$ and $b_i<b_{i+1}$).

\begin{enumerate}
\item[(a)] (80 pts)\ \ 
Given a sequence $\langle a_1,\ldots,a_n\rangle$ and an integer $k\le n-1$, we want
to compute a longest subsequence that has at most $k$ alternations.

(For example, for the input sequence $\langle 3,1,6,8,2,10,9,4,5,12,7,11\rangle$ and $k=2$,
an optimal subsequence is $\langle 1,6,8,10,9,4,5,7,11\rangle$, which has 2 alternations.)


Describe an $O(kn^2)$-time dynamic programming algorithm to solve this problem.
In this part, your algorithm only needs to output the optimal value (i.e., the length of the longest subsequence).
\item[(b)] (20 pts)\ \ 
Give pseudocode to also output an optimal subsequence.
\end{enumerate}


\bigskip
\item[Problem 6.2:] \
We have an $n\times 4$ grid, with $n$ rows and 4 columns.
We are given an $n\times 4$ matrix $F$, where $F[i,j]=1$ indicates that the grid cell
at the $i$-th row and $j$-th column is \emph{forbidden}, and $F[i,j]=0$ indicates that the cell is
``allowed''.
The goal is to cover the maximum number of grid cells using shapes of the following three types
(we are \emph{not} allowed to rotate these shapes):

\fig{hw6fig}{0.5,page=1}

The constraints are: (i)~no forbidden cells are covered, and (ii)~each cell
is covered at most once (i.e., the shapes can't overlap).

\newpage
In the following example with $n=8$, the forbidden cells are shaded in gray, and
the solution shown in red covers 22 cells, but is not optimal (can you do better?).

\fig{hw6fig}{0.5,page=2}

\vspace{-2ex}
\begin{enumerate}
\item[(a)] (90 pts)\ \
Design and analyze an efficient dynamic programming algorithm to solve this problem.
Your algorithm only needs to output the optimal value.

Hint: define a subproblem for each $i=1,\ldots,n$ and each of the 16 possible ``states'' that
the current row may be in\ldots

\item[(b)] (10 pts)\ \
If we change the problem to allow the shapes to be rotated (for example, the ``T'' shape can be rotated in 4 ways),
how would you change the definition of your subproblems, and how many subproblems
would you need as a function of $n$?
(For this part, don't give the recursive formula or the actual algorithm, since the details are messier.)

\end{enumerate}

\end{description}

\end{document}
