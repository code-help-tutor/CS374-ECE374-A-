\documentclass[11pt]{article}
\usepackage{fullpage,amsmath,hyperref}

\begin{document}

\begin{center}\Large\bf 
CS/ECE 374 A (Spring 2022)\\
{\Large Homework 1} (due Jan 27 Thursday at 10am)
\end{center}

\medskip\noindent\hrulefill\bigskip

\noindent
{\bf Instructions:} Carefully read\  
\url{http://engr.course.illinois.edu/cs374/sp2022/A/hw_policies.html}
and \url{http://engr.course.illinois.edu/cs374/sp2022/A/integrity.html}.


\begin{itemize}
    \item \textbf{Groups of up to three people may submit joint
       solutions.}  Each problem should be submitted by exactly one
    person, and the beginning of the homework should clearly state the
    Gradescope names and email addresses of each group member.  In
    addition, whoever submits the homework must tell Gradescope who
    their other group members are.

    \item \textbf{Submit your solutions electronically on the course
       Gradescope site as PDF files.}  Submit a separate PDF file for
    each numbered problem.  If you plan to typeset your solutions,
    please use the \LaTeX\ solution template on the course web site.
    If you must submit scanned handwritten solutions, please use a
    black pen on blank white paper and a high-quality scanner app (or
    an actual scanner, not just a phone camera).

    \item If you are not using your real name and your
    \texttt{illinois.edu} email address on Gradescope, you will need
    to fill in the form linked in the course web page. 

    \item \textbf{You may use any source at your disposal}---paper,
    electronic, or human---but you \emph{must} cite \emph{every} source
    that you use, and you \emph{must} write everything yourself in
    your own words.  

    \item Any homework or exam
    solution that breaks any of the following rules may be given an
    \emph{automatic zero}.
    \begin{itemize}
        \item Always give complete solutions, not just examples.
        \item Always declare all your variables, in English.  In
        particular, always describe the specific problem your
        algorithm is supposed to solve.     
        \item Always avoid unnecessarily long answers.
    \end{itemize}
\end{itemize}
\hrulefill

\newcommand{\eps}{\varepsilon}
\newcommand{\floor}[1]{\left\lfloor#1\right\rfloor}
\newcommand{\ceil}[1]{\left\lceil#1\right\rceil}


\newpage
\begin{description}
\item[Problem 1.1:]
Let $L\subseteq\{0,1\}^*$ be the language defined recursively 
as follows:
\begin{itemize}
\item The empty string $\eps$ is in $L$.
\item For any string $x$ in $L$,
the strings $0101x$ and $1010x$ are also in $L$.
\item For any strings $x,y$ such that $xy$ is in $L$, 
the strings $x00y$ and $x11y$ are also in $L$.
(In other words, inserting two consecutive 0's or two consecutive 1's anywhere to a string in $L$
yields another string in $L$.)
\item The only strings in $L$ are those that can be
obtained by the above rules.
\end{itemize}

Define $L_{ee}=\{ x\in\{0,1\}^*: \mbox{$x$ has an even number of 0's and an even number of 1's}\}$.
\begin{enumerate}
\item[(a)] Prove that $L \subseteq L_{ee}$, by using induction.  (You should use \emph{strong} induction.)
\item[(b)] Conversely, prove that $L_{ee}\subseteq L$, by using induction. 
\end{enumerate}

\bigskip
\item[Problem 1.2:]
Let $L=\{ x\in \{0,1,\ldots,9\}^*: \mbox{$x$ does not contain 374 as a substring}\}$.

Obviously, the number of strings in $\{0,1,\ldots,9\}^*$ of length $n$ is equal to $10^n$.

Prove that the number of strings in $L$ of length $n$ is at most $2\cdot 9.992^n$, by using induction.

[Hint: consider two cases: $x$ does not start with 3, or starts with 3.  In the second case, consider two subcases: the second symbol is not 7, or is 7.]

[We may give bonus points for a proof of an upper bound better than $O(9.990^n)$.]


\end{description}



\end{document}
