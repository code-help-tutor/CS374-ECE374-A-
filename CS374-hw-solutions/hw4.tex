\documentclass[11pt]{article}
\usepackage{fullpage,amsmath,hyperref}
\newcommand{\eps}{\varepsilon}

\begin{document}

\begin{center}\Large\bf 
CS/ECE 374 A (Spring 2022)\\
{\Large Homework 4} (due Feb 17 Thursday at 10am)
\end{center}

\medskip\noindent
{\bf Instructions:} As in previous homeworks.


\begin{description}
\bigskip
\item[Problem 4.1:]
For each of the following languages, determine whether it is regular or not, and give a proof.  To prove that a language is not regular, you should use the fooling set method.  (To prove that a language is regular, you are allowed to use known facts about regular languages, e.g., closure properties, all finite languages are regular, \ldots)

\begin{enumerate}
\item[(a)] $\{x (110)^n x^R: x\in \{0,1\}^*,\: n\ge 1\}$
\item[(b)] $\{0^i 1^j 0^k: \mbox{$i+k$ is divisible by 3, and $k$ is divisible by $j$, and $i,j,k\ge 1$}\}$
\item[(c)] $\{y x x^R z: x,y,z\in \{0,1\}^*,\ |x|\ge 374\}$ 
\item[(d)] $\{y0^n1^n0^nz: y,z\in \{0,1\}^*,\ n\ge 374\}$ 
\end{enumerate}


\bigskip
\item[Problem 4.2:]
Give a context-free grammar (CFG)
for each of the following languages.
You must provide explanation for how
your grammar works, by describing in English what is generated by
each non-terminal.  (Formal proofs of correctness are not required.)

\begin{enumerate}
\item[(a)] (30 pts)\ \ $\{x (110)^n x^R: x\in \{0,1\}^*,\: n\ge 1\}$
\item[(b)] (30 pts)\ \ $\{1^i 0^j 1^k: j=2i + 3k,\ i,j,k\ge 0 \}$
\item[(c)] (40 pts)\ \ $\{1^i 0^j 1^k: \mbox{$i+k$ is divisible by 3 and $0\le j\le k$}\}$ 
\end{enumerate}

\end{description}
\end{document}

