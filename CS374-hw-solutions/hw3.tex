\documentclass[11pt]{article}
\usepackage{fullpage,amsmath,hyperref}
\newcommand{\eps}{\varepsilon}

\begin{document}

\begin{center}\Large\bf 
CS/ECE 374 A (Spring 2022)\\
{\Large Homework 3} (due Feb 10 Thursday at 10am)
\end{center}

\medskip\noindent
{\bf Instructions:} As in previous homeworks.


\begin{description}
\bigskip
\item[Problem 3.1:]
For each of the following languages in parts (a), (b), and (c),
describe an NFA that accepts the language, using as few states as you can.
Provide a short explanation of your solution.
Below, $\#_0(x)$ and $\#_1(x)$ denote the number of 0's and the number of 1's in $x$ respectively.
\begin{enumerate}
\item[(a)] (30 pts)
all strings $x\in\{0,1\}^*$ such that ($x$ ends with 10101 or 11011) and ($\#_0(x)$ is divisible by 3 or $\#_1(x)$ is divisible by 3).
\item[(b)] (30 pts)
the language defined by the regular expression
$\big( ((01)^*0+2)(100)^*1 \big)^* \cdot (1^*+0^*2^*)$ over the alphabet $\{0,1,2\}$.
\item[(c)] (10 pts)
all strings in $\{0,1\}^*$ that contains the pattern $0?1?0$, where ``?''\ denotes ``don't care'' (i.e.,
a single symbol that is either 0 or 1);
in other words, the language defined by the regular expression $(0+1)^*\cdot 0(0+1)1(0+1)0\cdot (0+1)^*$.
\item[(d)] (30 pts)
Convert your NFA from part (c) to a DFA by using the subset construction
(i.e., power set construction).  
[{\em Note}: don't include unreachable states; also, several accepting states can be collapsed into one in this DFA.]
\end{enumerate}


\bigskip
\item[Problem 3.2:]
Given a language $L$ over the alphabet $\Sigma$,
define
\newcommand{\BACK}{\textsc{move-back}_8}
\[ \BACK(L) = \{xayz:\ xyaz\in L,\ x,y,z\in\Sigma^*,\ a\in\Sigma,\ |y|\le 8\}.
\]

Prove that if $L$ is regular, then $\BACK(L)$ is regular.

(For example, if $0100101001{\bf 1}0011\in L$, then
$01{\bf 1}001010010011\in \BACK(L)$.)%
\footnote{
\ldots and also $010{\bf 1}01010010011\in \BACK(L)$, and $0100{\bf 1}1010010011\in \BACK(L)$, \ldots,
$0100101001{\bf 1}0011\in \BACK(L)$.

For a different example: $\BACK(0^*1^*) = 0^*1^* + 0^*101^* + 0^*1001^* + 0^*10^31^*+ \cdots + 0^*10^81^*$.
}

[{\em Hint}: given an NFA (or DFA) for $L$, construct an NFA for $\BACK(L)$.
Give a formal description of your construction.  Provide an explanation of
how your NFA works, including the meaning of each state.  A formal proof
of correctness of your NFA is not required.]



\end{description}
\end{document}

